\documentclass{article}
\usepackage[spanish]{babel}
\usepackage{graphicx}
\usepackage{xcolor}
\usepackage[utf8]{inputenc}
\usepackage{fancyhdr}
\usepackage{lastpage}
\usepackage{enumitem}
\usepackage{listings}
\usepackage{float}
\usepackage{verbatim}

\pagestyle{fancy}
\fancyhf{}
\rfoot{Page \thepage\hspace{1pt} de~\pageref{LastPage}}

\title{Práctica Tika Apache}
\author{Guillermo López García}

\begin{document}
\maketitle

\textbf{Ejercicios.}
\begin{enumerate}
    \item El autor es `Lo'.
        \begin{figure}[H]
        \centering
        \includegraphics[width=0.7\linewidth]{./ej1}
        \caption{Ejercicio 1.}
        \end{figure}
    \item tika --text-main Tika.pdf $>$ Tika.doc
        \begin{figure}[H]
        \centering
        \includegraphics[width=0.7\linewidth]{./ej2}
        \caption{Ejercicio 2.}
        \end{figure}
    \item tika --metadata https://www.uca.es
        \begin{figure}[H]
        \centering
        \includegraphics[width=0.7\linewidth]{./ej3}
        \caption{Ejercicio 3.}
        \end{figure}
    \item
        \begin{enumerate}
            \item echo `Archivo de texto plano listo para comprimir.' $>$ archivo.txt
            \item cat archivo.txt
            \item zip archivo.zip archivo.txt
            \item tika --detect archivo.zip
                \begin{figure}[H]
                \centering
                \includegraphics[width=0.7\linewidth]{./ej4}
                \caption{Ejercicio 4.}
                \end{figure}
            \item tika --metadata archivo.zip
                \begin{figure}[H]
                \centering
                \includegraphics[width=0.7\linewidth]{./ej5}
                \caption{Ejercicio 4.}
                \end{figure}
        \end{enumerate}
    \item
        \begin{enumerate}
            \item tika --metadata ImagenCorreo.jpg
                \begin{figure}[H]
                \centering
                \includegraphics[width=0.7\linewidth]{./ej6}
                \caption{Ejercicio 5.}
                \end{figure}
            \item tika --metadata ImagenWhatsapp.jpeg
                \begin{figure}[H]
                \centering
                \includegraphics[width=0.7\linewidth]{./ej7}
                \caption{Ejercicio 5.}
                \end{figure}
        \end{enumerate}
    \item tika --metadata https://www.uca.es $>$ metadatosuca.txt
    \verbatiminput{metadatosuca.txt}
    \item
        \begin{enumerate}
            \item tika --text-main celebs\_fixed.rdf $>$ celebs\_fixed.doc
                \begin{figure}[H]
                \centering
                \includegraphics[width=0.7\linewidth]{./ej9}
                \caption{Ejercicio 7.}
                \end{figure}
            \item tika --text-main celebs\_fixed.rdf $>$ celebs\_fixed.pdf
                \begin{figure}[H]
                \centering
                \includegraphics[width=0.7\linewidth]{./ej10}
                \caption{Ejercicio 7.}
                \end{figure}
            \item okular celebs\_fixed.pdf \&
                \begin{figure}[H]
                \centering
                \includegraphics[width=0.7\linewidth]{./ej11}
                \caption{Ejercicio 7.}
                \end{figure}
        \end{enumerate}
    \item
        \begin{enumerate}
            \item tika --metadata q.jpg $>$ q.txt
            \item tika --metadata r.jpg $>$ q.txt
            \item tika --metadata s.JPG $>$ q.txt
        \end{enumerate}
        \begin{figure}[H]
        \centering
        \includegraphics[width=0.7\linewidth]{./ej12}
        \caption{Ejercicio 8.}
        \end{figure}
        \begin{figure}[H]
        \centering
        \includegraphics[width=0.7\linewidth]{./ej13}
        \caption{Ejercicio 8.}
        \end{figure}
    \item
        \begin{enumerate}
            \item tika --html https://www.uca.es $>$ uca.html
                \begin{figure}[H]
                \centering
                \includegraphics[width=0.7\linewidth]{./ej14}
                \caption{Ejercicio 9.}
                \end{figure}
            \item tika --html uca.html $>$ uca.doc
                \begin{figure}[H]
                \centering
                \includegraphics[width=0.7\linewidth]{./ej15}
                \caption{Ejercicio 9.}
                \end{figure}
            \item tika --metadata uca.doc
                \begin{figure}[H]
                \centering
                \includegraphics[width=0.7\linewidth]{./ej16}
                \caption{Ejercicio 9.}
                \end{figure}
        \end{enumerate}
    \item
        \begin{enumerate}
            \item tika --metadata primera.jpg $>$ primera.txt
                \begin{figure}[H]
                \centering
                \includegraphics[width=0.7\linewidth]{./primera}
                \caption{Ejercicio 10.}
                \end{figure}
                \verbatiminput{primera.txt}
            \item tika --metadata segunda.png $>$ segunda.txt
                \begin{figure}[H]
                \centering
                \includegraphics[width=0.7\linewidth]{./segunda}
                \caption{Ejercicio 10.}
                \end{figure}
                \verbatiminput{primera.txt}
            \item tika --metadata tercera.png $>$ tercera.txt
                \begin{figure}[H]
                \centering
                \includegraphics[width=0.7\linewidth]{./tercera}
                \caption{Ejercicio 10.}
                \end{figure}
                \verbatiminput{primera.txt}
        \end{enumerate}
    \item Los pasos a seguir son:
        \begin{enumerate}
            \item Instalar eclipse.
            \item Crear un proyecto en Java con eclipse.
            \item Ir a la zona de `Project $>$ Properties $>$ Java Build Paths $>$ Libraries $>$ Add External JARS\ldots'.
                \begin{figure}[H]
                \centering
                \includegraphics[width=0.7\linewidth]{./ej18}
                \caption{Ejercicio 11.}
                \end{figure}
            \item Seleccionar el jar de la app de tika descargado de la web.
            \item Crear un archivo Main.java donde crearemos el código relevante.
                \begin{figure}[H]
                \centering
                \includegraphics[width=0.7\linewidth]{./ej19}
                \caption{Ejercicio 11.}
                \end{figure}
            \item Guarda el código relevante.
            \item Ejecutar el código relevante.
                \begin{figure}[H]
                \centering
                \includegraphics[width=0.7\linewidth]{./ej20}
                \caption{Ejercicio 11.}
                \end{figure}
        \end{enumerate}
        \lstset{language=Java, texcl=true}
        \begin{lstlisting}[frame=single]
import java.io.FileInputStream;
import java.io.IOException;
import java.io.InputStream;

import org.apache.tika.metadata.Metadata;
import org.apache.tika.parser.AutoDetectParser;
import org.apache.tika.parser.ParseContext;
import org.apache.tika.sax.BodyContentHandler;

public class Main
{
     public static void main(String args[])
     throws Exception
     {
          Main tika = new Main();
          tika.convertPdf("/home/oem/tika/Tika.pdf");
     }
     
     public void convertPdf(String fileName)
     {
          InputStream stream = null;
          try
          {
              stream = new FileInputStream(fileName);
              AutoDetectParser parser =
              new AutoDetectParser();
              BodyContentHandler handler =
              new BodyContentHandler(-1);
              Metadata metadata = new Metadata();
              parser.parse(
                stream,
                handler,
                metadata,
                new ParseContext()
              );
              System.out.println(handler.toString());
          } catch (Exception e)
          {
              e.printStackTrace();
          } finally
          {
              if(stream!=null)
                   try
                   {
                        stream.close();
                   } catch (IOException e)
                   {
                        System.out.println(
                            "Error closing stream"
                        );
                   }
          }
     }
}
        \end{lstlisting}
\end{enumerate}

\end{document}
